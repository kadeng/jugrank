\documentclass[a4paper]{article}

\usepackage[english]{babel}
\usepackage[utf8]{inputenc}
\usepackage{amsmath}
\usepackage{amssymb}                                                
\usepackage{amsfonts}
\usepackage{framed} 
\usepackage{bm}
\usepackage{graphicx}
\usepackage[T1]{fontenc}
\usepackage[colorinlistoftodos]{todonotes}

\title{JRank, a new Ranking and Tournament System for competitive Games}
\author{Kai Londenberg (Kai.Londenberg@gmail.com)}
\date{2010-08-25}

\begin{document}

\maketitle

\begin{abstract}

Tournaments of a size where it gets impractical to let every player or team play against each other usually employ tournament and ranking systems which are
far from being optimal in terms of efficiency or accuracy in their ranking results. JRank, a new tournament and ranking system is presented and analyzed, which
seeks to eliminate many known deficiencies in existing tournament and ranking systems. It consists of an iterative algorithm for calculating a numerical 
\textit{strength} for each player and of a tournament system which enhances on the known swiss system. It is shown that the algorithm converges to a solution.
A method for adapting the methodology to different types of games by optimizing parameters for the minimization of errors on historical tournament data is given.
Subsequently, the tournament system is evaluated in a monte carlo simulation, in which it's performance is compared with other popular tournament
and ranking systems. The simulations show that the new system seems to outperform popular existing tournament systems in terms of accuracy and efficiency by 
a wide margin. The paper closes with a discussion about possible statistical interpretations of the obtained results and possible variations o the system.

\end{abstract}

\section*{Introduction}



\section*{Ranking Algorithm}

Let P be the number of players, where each player is denoted by an index c, with $ c \in \mathbb{N} ; 1 <= p <= C $. Furthermore, let $ G _ p $ be the number of
all games played by player p, each game denoted by the index g, with$ g \in \mathbb{N} ; 1 <= g <= G_p $. Formula elements with a subscript of pg
are therefore associated with the player p and game g of that player. 
\newline
\newline
Let $ W _ p \in \mathbb{R^*_+} $ be the absolute \textit{strength} of player p, and 
$ O _{pg} \in \mathbb{R^*_+} $ be the \textit{strength} of the opposing player in the game g of player p. 
\newline
\newline
For each game, we have two scores: $ A_{pg} \in \mathbb{N} $ is the number of points scored by player p in game g, while $ B_{cpg} \in \mathbb{N} $
is the number of points scored by the opposing player.  
\newline
\newline
Every game of one player against another can be interpreted to be a statistical measure of the relative \textit{strengths} of both players. Therefore, we have

\begin{equation}
 \frac{W_p}{ O_{pg} } = \frac{M(A_{pg})}{M(B_{pg})}  
\end{equation}

with $ M : \mathbb{N} \mapsto \mathbb{R^*_+} $ being a function which models the possibly nonlinear relationship between the ratio of the strengths, and
the game results. One possible and reasonable definition of M(x) is

\begin{equation}
M(x) = ( x + k)^p
\end{equation}

with $ k,p \in \mathbb{R*_+} $ being model parameters which may be used to adapt for different game characteristics.
\par

Solving (1) for $ W_p $, we get

\begin{equation}
 W_p = O_{pg} \frac{M(A_{pg})}{M(B_{pg})}  
\end{equation}

To average out measurement errors, we take the geometric average of all measurements (i.e. Games), leading to

\begin{equation}
 W_p = \sqrt[G]{\displaystyle\prod_{g=1}^{G}{O_{pg}\frac{M(A_{pg})}{M(B_{pg})}}}
\end{equation}

Therefore, we have a system of P nonlinear equations with multiple variables ( $ W_p $ ) to be solved. 
In the next section it will be shown that this can be done iteratively by choosing arbitrary
$ W_p(0) > 0 $ and calculating the sequence:

\begin{equation}
W_p(n+1) = \sqrt[G]{\displaystyle\prod_{g=1}^{G}{O_{pg}(n)\frac{M(A_{pg})}{M(B_{pg})}}}
\end{equation} 

\pagebreak


\subsection*{Proof of Convergence}

To prove the convergence of these successive approximations, a suitable convergence criterion is available in
literature about basic numerical analysis. The following box contains quotes from Rainer Kress, Numerical Analysis \cite[p.98ff]{NumAna}
\small
\begin{framed}

\textbf{Convex Subset} A subset D of a linear space X is called convex if $ \lambda x + ( 1 - \lambda ) y \in D $ for all $ x,y \in D $ and all $ \lambda \in (0,1)$, i.e., if the
straight line connecting x and y is contained in D. 
\newline \newline
[...]
\newline \newline
\textbf{Jacobian} 

Let f : $ D \mapsto \mathbb{R}^n $ be a mapping

$$
f(x) = (f_1(x_1, ..., x_n), ..., f_n(x_1,...,x_n))^C
$$

where the $ f_j : D \mapsto \mathbb{R}, j = 1, ..., n $ are continuously differentiable functions. By 

$$
f'(x) = \left( \frac{\partial f_j}{\partial x_k}  \right)_{j,k=1,...n}
$$

we denote the Jacobian matrix of f.
\newline \newline
[...]
\newline \newline
\textbf{Theorem 6.8} Let $ D \subset \mathbb{R}^n $ be closed and convex (with a nonempty interior) and let $ f : D \mapsto D $ be a continuous mapping. Assume further that f is
continuously differentiable in the interior of D and that it's Jacobian can be continuously extended to all of D such that

$$
\sup_{x \in D} \Vert f'(x)\Vert < 1
$$

in some norm $ \Vert \cdot \Vert $ on $ \mathbb{R}^n $. Then the equation f(x) = x has a unique solution $ x \in D$, and the successive approximations 

$$
 x_{v+1} := f(x_v),  v = 0,1,2,...
$$

converge for each $ x_0 \in D $ to this fixed point. We have the a priori error estimate 

$$
\Vert x_v - x \Vert \leq \frac{q^v}{1-q} \Vert x_1 - x_0 \Vert
$$
and the a posteriori error estimate
$$
\Vert x_v - x \Vert \leq \frac{q}{1-q} \Vert x_v - x_{v-1} \Vert
$$
for all $ v \in \mathbb{N} $

\end{framed}


In the following, it will be shown that the system of nonlinear equations (4) fulfills the criterions laid out above, therefore, the set of iterative
sequences (5) converges to a solution, and the error estimates given above may be applied.
\newline\newline\newline
For a given set of players and games, $R_{pg}$, $R_{min}$ , $R_{max}$ are defined as a specific game result for player p, game g, and the worst and best results respectively: 
\newline \newline
\begin{center}
$ R_{pg} = \frac{M(A_{pg})}{M(B_{pg})} $, $ R_{min} = \displaystyle\min(\frac{M(A_{pg})}{M(B_{pg})}) $, $R_{max} = \displaystyle\sup(\frac{M(A_{pg})}{M(B_{pg})}) $
\end{center}

Equation (5) may be written as 

\begin{equation}
W_p(n+1) = \sqrt[G]{\displaystyle\prod_{g=1}^{G}{O_{pg}(n)R_{pg}}} 
\end{equation} 

Which obviously has an upper bound in $$\displaystyle\sup_{1 \leq p \leq P}(W_p(n+1)) \leq \displaystyle\sup_{1 \leq p \leq P}(W_p(n))R_{max} $$
Therefore, we have an upper bound for the sequence, given by $$ W_{max} = \displaystyle\sup_{1 \leq p \leq P}(W_p(0)) R_{max} $$
Likewise, we have a lower bound for the sequence, given by $$ W_{min} = \displaystyle\min_{1 \leq p \leq P}(W_p(0)) R_{min} $$
\newline\newline

This allows to formulate the sequence in terms of a continuous mapping of a closed and convex subset of $\mathbb{R}^P$ required by the convergence Theorem 6.8.
\newline\newline
Let $ D = ( W_{min}, W_{max} )^P \subset \mathbb{R}^P $ with $ W = (W_1, ..., W_p) \in D$. Then \textbf{D is convex}, since componentwise the following statement can
be trivially proven to be true $$ \lambda x_p + ( 1 - \lambda ) y_p \in ( W_{min}, W_{max} )$$ for all $ 1 \leq p \leq P$, all $ x,y \in D $ and all 
$ \lambda \in (0,1)$. Likewise, \textbf{D is closed} since it consists componentwise of closed intervals over $ \mathbb{R} $,
which are known to be closed sets, and each union of closed sets is again a closed set.  
\newline\newline
Furthermore, let f be a mapping where, componentwise we have

\begin{equation}
f_p(W) = \sqrt[G]{\displaystyle\prod_{g=1}^{G}{O_{pg}(n)R_{pg}}} 
\end{equation}

The proof for upper and lower bounds for the sequence $ W_t $ can be likewise applied to $ f_p(W) $, so 
$ f_p(W) \in (W_{min},W_{max})$ for each $W \in D$, therefore
$ f(W) \in D $ for each $ W \in D$, so f is in fact a mapping $ f : D \mapsto D $.
\newline\newline
Remember that $ O_{pg} $ is the strength of the opponent in game g of player p.  For each p, g  we have a $ t
\in \mathbb{N}, t \neq p$ so that $ O_{pg} = W_t $, therefore f is indeed a mapping 

$$
f(W) = (f_1(W_1, ..., W_p), ..., f_p(W_1,...,W_p))^T
$$

which is obviously continuously differentiable in all of D.

\pagebreak
It is not required that each player plays against each other and a player can't play against himself.
The Jacobian matrix is therefore zero for each combination of players which did not play against each other:
\newline\newline
\begin{center}
$ f'_{pt}(W) = \frac{\partial f_p}{\partial W_t}(W)  = 0 $  for each p which did not play against t 
\end{center}

For all other cases, we have a partial derivative of $ f_p(W) $ 

\begin{equation}
f'_{pt}(W) = \frac{\partial f_p}{\partial W_t}(W) = \frac{1}{W_t}\sqrt[G]{\displaystyle\prod_{g=1}^{G}{O_{pg}(n)R_{pg}}} 
\end{equation}

Where $ W_t $ is equal to $ O_{pg} $ for one combination of p and g. This can be written as 

$$
f'_{pt}(W) = \frac{\partial f_p}{\partial W_{t}}(W) = \frac{1}{W_t} f_p(W)
$$

which obviously has an upper bound

\begin{equation}
\sup_{W \in D} f'_{pt}(W) = \frac{W_{max}}{W_{min}} 
\end{equation}

Which, using the $ L_{\infty} $ norm can be extended to the entire Jacobian.
\newline\newline
This allows to choose a norm $ \Vert A \Vert _D =   \frac{W{min}}{{W{max}+1}} \Vert A \Vert _{\infty} $ for which, obviously 
$$
\sup _{x \in D} \Vert f'(x) \Vert _D < 1
$$

Therefore, all convergence criterions are met. It can be concluded that the sequences of equation (5) converge on the solution
to the nonlinear equation system (4). We also have an error estimate which can be used to determine a maximum error at step n. 
Therefore, the solution to equation (4) can be calculated to any desired accuracy using a finite number of steps.

\subsection*{Adapting the ranking algorithm to real world games}

The function $ M : \mathbb{N} \mapsto \mathbb{R^*_+} $ models the possibly nonlinear relationship between the expected ratio of points
between two players and their respective strengths. Recalling equation (1), we have

$$
 \frac{W_p}{ O_{pg} } = \frac{M(A_{pg})}{M(B_{pg})}  
$$

It would be desirable to be able to adapt M to different game characteristics. Games like Table Tennis, Jugger, Badminton, 
Basketball or Soccer each have vastly different characteristics when it comes to the number of points which are usually achieved within a game,
randomness of results and influence on the relative strengths of the players (or teams, respectively). Also, there might be tournaments or leagues
which have a very homogenous field of players (i.e. the players are similar in strength), where other tournaments or leagues might be have a high variance in
the strengths of the players.
\newline
\newline
For most games, there are vast amounts of historical game results avaiable which may be used to optimize the model.
Given a set of historical game results, it is possible to use the previously specified ranking algorithm to calculate strengths for all participating teams given
a model function M. Using these calculated strengths, we can calculate expected ratios of points. The difference between these can be interpreted as the 
error $ E_g $ for a given game g with results $A_g:B_g$ of the players with the strengths $W_g$ (which scored $A_b$) and $O_g$ (which scored $B_g$) respectively:

\begin{equation}
E_g(M) = \left\vert \frac{W_g}{ O_g } - \frac{M(A_{g})}{M(B_{g})} \right\vert  
\end{equation}
\newline\newline
Therefore, the choice of a suitable M is reduced to an optimization problem which consists of finding 
a suitable type of model function, and adapting model parameters to minimize errors. The most common way
to achieve this would be to minimize the sum of squared errors - i.e. Least Squares method. 
\newline\newline
Other definitions of error might be more useful and lead to better results, for example it might be sensible to minimize the number of incorrectly
predicted winners with topmost priority, while trying to reduce prediction errors in the achieved point ratios is only of secondary
importance. The error function and optimization method could be adapted to reflect this.  
\newline\newline
One possible choice for M(x) has already been given in equation (2):

$$
M(x) = ( x + k)^p
$$

where $ k,p \in \mathbb{R} $ are model parameters. 
\newline\newline

By choosing reasonable ranges for the model parameters, it is now easily possible to find near-optimal model parameters by performing a complete
grid search over k and p, and choosing those values of k and p which minimize the error and best meet the desired criterions.
\newline\newline
For the function given above, reasonal default ranges for k and p which are suitable for most games would probably be 
$ 0.001 \leq k \leq 20 $ and $ 0.001 \leq  p \leq 3 $.
\newline\newline
Without any adaptation, k=1 and p=0.5 seem to be reasonable default values.

\subsection*{Statistical Interpretation and Analysis}

So far it has been assumed that a game result as given by the points achieved by two opposing players is indeed a statistical measure of their
relative strengths. If we also assume that we have found an optimal model by adapting it to real world historical game data, as outlined
in the previous section, it would be reasonable to assume that the remaining errors as given by equation (7) have random cause, and therefore
follow a random distribution centered at 0.

This is an assumption that occurs very frequently and can therefore be verified or rejected using standard statistical procedure.


\begin{thebibliography}{9}
\bibitem[Rainer Kress, Numerical Analysis]{NumAna}
  Rainer Kress,
  \emph{Numerical Analysis}.
  Springer, ISBN 0-387-98408-9
\end{thebibliography}

\end{document}